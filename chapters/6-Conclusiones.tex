\chapter{Conclusiones}
\label{cap:conclusiones}
En este trabajo se ha propuesto la generaci\'on de un sistema el\'astico, cuyo objetivo es lograr una mejor utilizaci\'on de los recursos disponibles \normalsize{en un sistema} y con ello un aumento en su capacidad de procesamiento y escalabilidad.

\section{Detalles de la contribuci\'on}
Dentro de las contribuciones de este trabajo se encuentra el dise\~no e implementaci\'on de un modelo el\'astico que es capaz de lidiar con el dinamismo del flujo de eventos o tr\'afico de datos. En este modelo se dise\~naron cuatro m\'odulos, los cuales estaban compuestos por un m\'odulo de monitoreo, que recolecta las estad\'isticas, un m\'odulo reactivo y predictivo, que estima la carga de cada operador en el presente y a futuro respectivamente, y un m\'odulo de administraci\'on de r\'eplicas, que aumenta o disminuye el n\'umero de r\'eplicas de un operador seg\'un la carga de \'este.

El m\'odulo reactivo que se ha dise\~nado e implementado tiene como funci\'on analizar la carga actual del operador, cumpliendo as\'i el primer objetivo de este trabajo, el cual consiste en dise\~nar e implementar un algoritmo reactivo que permita analizar en el momento la carga de los operadores.

Por otra parte, se ha dise\~nado e implementando el m\'odulo predictivo, cuya funci\'on es estimar la carga de un operador en una ventana de tiempo futura, cumpliendo el segundo objetivo, el cual consiste en dise\~nar e implementar un algoritmo de predicci\'on que permita estimar la carga de los operadores.

As\'i mismo, se ha dise\~nado e implementado un m\'odulo de administraci\'on de r\'eplicas, el cual se encarga de administrar la cantidad de r\'eplicas de los operadores del SPS de forma el\'astica, vale decir, que aumenta o disminuye el n\'umero de r\'eplicas acorde al tr\'afico recibido.

%Por otra parte, se han dise\~nado y construido distintos experimentos que permita validar la hip\'otesis planteada, cuyo planteamiento es la utilizaci\'on de modelo el\'astico de tal manera que mejore el rendimiento del SPS y se procese mayor cantidad de eventos, cumpliendo as\'i el cuarto objetivo planteado.

Para validar el modelo el\'astico se \normalsize{han} construido tres escenarios para la experimentaci\'on, donde el \normalsize{primero} consiste en una aplicaci\'on que realiza operaciones sin estados, la segunda una aplicaci\'on que realiza operaciones con estados, y la tercera una aplicaci\'on sint\'etica. De esta manera, el objetivo es evaluar el rendimiento del sistema utilizando el modelo el\'astico.

%Y finalmente, se ha evaluado y analizado el rendimiento del sistema a trav\'es de distintas aplicaciones generadas sobre un SPS, en este caso sobre S4. En todas las aplicaciones ha mejorado la cantidad de eventos procesados, dependiendo del tipo de aplicaci\'on se ha detectado un aumento de hasta 8 veces m\'as eventos procesados. Como se hab\'ia mencionado anterior, al posee mayor cantidad de datos procesados, se posee mayor precisi\'on en la informaci\'on obtenida. Por otra parte, el costo asociado por la implementaci\'on del modelo el\'astico es de un aumento de $0,0119\%$, pero con una disminuci\'on del uso de la memoria RAM, la cual es de un $1,5187\%$, lo cual significa que el sistema si bien puede aumentar el consumo de CPU, disminuye el uso de la RAM.

Haciendo uso de los escenarios, se ha evaluado y analizado el rendimiento del sistema con y sin uso del modelo el\'astico. Los resultados muestran que para todos los escenarios se ha mejorado la cantidad de eventos procesados. Dependiendo del tipo de escenario, se ha detectado un aumento de hasta 9 veces en el \textit{throughput} de la aplicaci\'on. Este es un resultado importante, puesto que al poseer una mayor cantidad de datos procesados, se puede lograr una mayor precisi\'on en la informaci\'on obtenida. Por otra parte, el costo asociado a la implementaci\'on del modelo el\'astico en relaci\'on a la CPU es de un aumento del $0,01\%$, sin embargo la memoria RAM utilizada ha disminuido en un $1,51\%$.

%Por lo que no s\'olo se lograron todos los objetivos planteados, sino que se ha demostrado la hip\'otesis planteada en el inicio del trabajo, donde seg\'un los distintos experimentos realizados, se ha determinado que el modelo cumple con una mejora en el SPS, aumentando la cantidad de datos procesados. Pero adem\'as de esto, se ha concluido que el sistema posee un bajo costo de implementaci\'on, adem\'as de una ganancia en el consumo de memoria.

De esta manera, podemos ver que se han alcanzado todos los objetivos planteados, y la hip\'otesis establecida se ha validado. El modelo cumple con hacer el\'astico el SPS, aumentando la cantidad de datos procesos acorde al tr\'afico recibido. Adem\'as, se ha concluido que el sistema posee un bajo costo de implementaci\'on.

\section{Discusiones}

%Uno de los problemas detectados en el desarrollo de este trabajo es la implementaci\'on realizada en S4, debido que el SPS posee ciertas falencias para la implementaci\'on del modelo dise\~nado. Esto se debe a que la cantidad de eventos entrantes no son todos procesados, independientemente si se genera una mayor cantidad de r\'eplicas o no. Esto fue detectado en la fase de experimentaci\'on, donde al tratar de realizar pruebas con un tiempo de ejecuci\'on mayor, existe una disminuci\'on considerablemente de la tasa de rendimiento de un operador despu\'es de un largo tiempo de ejecuci\'on, debido a que el \textit{buffer} del operador se llena al no procesar todos los datos entrantes, bloqueando el env\'io de eventos a \'este.

\normalsize{Uno de los problemas detectados en el desarrollo de este trabajo es la limitaci\'on de recursos f\'isicos para la implementaci\'on de la aplicaci\'on dise\~nada.} Esto se debe a que la cantidad de eventos entrantes no son todos procesados, independientemente si se genera una mayor cantidad de r\'eplicas o no. Este problema fue detectado en la fase de experimentaci\'on, donde al tratar de realizar pruebas con un tiempo de ejecuci\'on mayor, existe una disminuci\'on considerablemente de la tasa de rendimiento de un operador despu\'es de un largo tiempo de ejecuci\'on, debido a que el \textit{buffer} del operador se llena al no procesar todos los datos entrantes, bloqueando el env\'io de eventos a \'este.

%Dentro de las limitaciones del trabajo est\'a la homogeneidad de la tasa de procesamiento. Como se hab\'ia presentando en las limitaciones, este modelo asume una tasa de procesamiento homog\'enea para cada uno de los eventos entrantes, la cual es obtenida, en general, en las primeras ventanas de tiempo de la ejecuci\'on. Esto es importante de considerar, porque de ser heterog\'eneos los eventos entrantes, se calculan tasas de procesamientos las cuales no garantizan que los pr\'oximos eventos posean la misma tasa de procesamiento. Puede darse el caso que su tasa de procesamiento sea m\'as alta o m\'as baja, por lo que el c\'alculo de la tasa de rendimiento es err\'oneo, generando una mala estimaci\'on de la carga del operador.

Dentro de las limitaciones del trabajo est\'a \normalsize{el c\'alculo de la tasa de procesamiento, la cual se considera homog\'enea en todo en transcurso de la ejecuci\'on del sistema. Esto significa que se calcula un valor al principio de la ejecuci\'on del sistema, el cual indica cual es la cantidad de eventos que procesa por segundo, de tal manera de considerar esa tasa de procesamiento en los c\'alculos de la tasa de rendimiento de los operadores. En caso que no se considere esto, se tiene que calcular una tasa de procesamiento en ventanas de tiempo, lo cual puede producir un porcentaje de error, debido que la tasa de procesamiento de la ventana de tiempo anterior no sea la misma que la actual.}

Si bien el modelo dise\~nado realiza un an\'alisis del sistema l\'ogico, no considera un an\'alisis de los recursos f\'isicos disponibles por parte del sistema, es decir, uso de la CPU, capacidad de la memoria RAM, entre otras m\'etricas. Cabe destacar, que en el caso que la carga de un operador aumente en conjunto con la replicaci\'on, existe la limitaci\'on f\'isica de la m\'aquina, debido a la capacidad de la CPU y memoria que \'este posea, limitando la capacidad de procesamiento del SPS.

Por otra parte, el sistema no es capaz de detectar patrones estacionarios que puedan existir en el d\'ia, lo cual es una desventaja en la implementaci\'on de \'este. Esto se debe a que el algoritmo predictivo analiza procesos estoc\'asticos, y no un aprendizaje del comportamiento del flujo de datos, como lo realizan as\'i modelos predictivos tipo \textit{machine learning} \citep{bookMohri2012}. Sin embargo, estas soluciones son de alto costo y limitan la escalabilidad del SPS.

Finalmente, el sistema dise\~nado presenta como ventaja poseer un bajo c\'omputo para el c\'alculo del n\'umero de r\'eplicas. Tambi\'en se destaca el r\'apido an\'alisis de los operadores, ya sea en la distribuci\'on de carga en cada uno de los operadores, o en el estado que se encuentra el operador, de tal manera de modificar la cantidad de r\'eplicas existentes. De esta manera, al poseer un sistema el\'astico, se logra optimizar el uso de los recursos existentes y adquirir un dinamismo en el grafo de la aplicaci\'on ejecutada sobre el SPS.

\section{Trabajo futuro}
%Dentro de las mejoras que se puede realizar al sistema son fundamental tres: un modelo el\'astico que trabaje con m\'as de un nodo f\'isico, un predictor que indique din\'amicamente cuantas son las r\'eplicas necesarias seg\'un el historial y la implementaci\'on del sistema dise\~nado en otro SPS.

Dentro de los problemas abiertos que pueden resolverse a futuro en el modelo propuesto, se encuentran: el dise\~no de un predictor que indique din\'amicamente cu\'antas son las r\'eplicas necesarias seg\'un el historial y la implementaci\'on del modelo dise\~nado en otro SPS.

%En el primer caso, se podr\'ia realizar un sistema de monitoreo, en el cual se posea una m\'aquina para analizar los datos de cada una de las m\'aquinas disponibles, y \'esta posea adem\'as las r\'eplicas primarias de cada operador. En caso que exista una sobrecarga por parte de un operador, es necesario realizar una r\'eplica por parte del monitor centralizado, y que este determine a cual de las m\'aquinas disponibles debe enviar los datos seg\'un la cantidad de recursos disponibles, tasa de procesamiento por parte del operador, entre otras variables. De esta manera, se posee un sistema escalable, debido que se puede implementar un SPS en un servicio de \textit{Cloud Computing}, de tal manera que en caso que sea necesario mayor cantidad de m\'aquinas, se a\~nadan y el monitor pueda distribuir mayor carga a estas nuevas m\'aquinas, en caso de ser requerido.

Actualmente el modelo propuesto asume de manera est\'atica un n\'umero fijo de r\'eplicas a generar, se podr\'ia realizar un an\'alisis m\'as detallado de la historia, de tal manera que seg\'un el comportamiento que \'este posea, estimar cuantas ser\'ian las r\'eplicas a aumentar o disminuir. As\'i mismo, se podr\'ia realizar un estudio respecto a los \textit{peaks} de tr\'afico que se encuentren de forma estacionaria seg\'un cierto flujo de datos, y que el sistema se adapte din\'amicamente a estos. Para realizar estos estudios, se puede a\~nadir la capacidad de aprendizaje utilizando \textit{machine learning} \citep{bookMohri2012} al modelo propuesto. De esta manera, \'este aprender\'ia los comportamientos respectivos del tr\'afico, de manera de mejorar la predicci\'on y con ello la utilizaci\'on de recursos.

%Por ejemplo, en el caso de \textit{Twitter} existen per\'iodos del d\'ia que los usuarios comentan m\'as, por lo tanto, en esos per\'iodos aumentar la cantidad de recursos, y en los per\'iodos que no comentan tanto, se podr\'ia disminuir la cantidad de recursos, por lo que se podr\'ia evaluar alternativas para el predictor como \textit{machine learning} \citep{bookMohri2012}.

Por otra parte, ser\'ia interesante poder implementar el modelo el\'astico en otro SPS, ya sea Storm \citep{stormtwitter} o StreamIt \citep{ThiesKA02}, debido a los distintos problemas que surgieron al utilizar el S4. De esta manera, se podr\'ia realizar una comparaci\'on de cual son los distintos pro y contra de los SPS con el sistema implementando, y en que casos es mejor utilizar uno u otro dependiendo del modelo o escenario que estos utilicen.