\chapter{Introducción}
\label{cap:introduccion}

\section{Antecedentes y motivación}
\label{intro:motivacion}

La gran contribución de información en la Internet se ha debido al origen de la Web 2.0, donde ésta se caracteriza por la participación activa del usuario, siendo reflejado en el auge de blogs, redes sociales u otras aplicaciones web \citep{web2007oberhelman}. Con el objetivo de extraer información de dichos datos, se crean sistemas de procesamiento para grandes cantidades de información generadas por la interacción entre los usuarios.

Con el paso del tiempo, más y más información es generada por distintas interacciones generadas por los usuarios. Por lo que, analizar o extraer esta información no es una tarea fácil, más aún cuando muchas de estas interacciones deben ser analizadas en tiempo real, dada su dependencia temporal. Por lo que sistemas tradicionales de procesamiento basados en MapReduce \citep{2010Lin} o \textit{bash processing} \citep{HawwashN14} no son los ideales para analizar esta información.

Es así como con el tiempo se han ido creando distintas aplicaciones de procesamiento en tiempo real, debido al interesante funcionamiento que poseen, las que se caracterizan por ser capaces de procesar grandes flujos de datos en tiempo real \citep{ChenZ14a}. La necesidad de procesar informaci\'on en tiempo real surge dado que muchas aplicaciones, donde sus usuarios requieren de respuestas r\'apidas y actualizadas que le permitan tomar decisiones en per\'iodos cortos de tiempo. Dentro de los ejemplos existentes se encuentran; análisis de sentimientos de los mensajes de usuarios, análisis de los precios de la bolsa de valores, recopilación de información en caso de emergencia, entre otros. Las distintas aplicaciones que se han creado se volvieron críticas para sus usuarios, debido que sustenta la toma de decisiones de empresas o instituciones \citep{Wenzel14}.

Un ejemplo de esto, son las aplicaciones que analizan redes sociales en caso de un desastre natural, donde grandes cantidades de información son generadas, y se requiere procesar esta información lo más cercano al tiempo real para obtener información que sea relevante para la situación \citep{andrade2014fundamentals}. De esta manera, se puede construir un sistema que pueda procesar los datos realizando análisis de sentimiento, búsqueda de palabras claves o filtros de búsqueda, ya sea por idioma, país o género. Con esta información, se puede establecer sectores críticos, facilitar la búsqueda de personas, distribución de alertas, o detección de necesidades, lo cual sería crucial para tomar decisiones en esos momentos.

Por otra parte, también son utilizado estos sistemas de procesamiento para llevar a cabo predicciones en la bolsa de comercio, de esta manera, se crean sistemas de procesamiento que apliquen modelos matemáticos y permiten predecir el comportamiento para el siguiente día en el mercado. Con estos sistemas, la ganancia que existe por parte de las personas interesadas puede aumentar considerablemente, por lo que ha generando un alto interés en el desarrollo e investigación en esta área.

También se aplica en casos de seguridad, dado que se realiza un monitoreo de la actividad que surge por parte de los usuarios que interactúan en una red específica. Esto es útil para empresas o ministerios que poseen información privilegiada, y en caso que alguien desee realizar respaldos o eliminar información sin consentimiento de los encargados, puede detectarse la persona y generarse una alarma de preventiva a las autoridades. Como la información es procesada en tiempo real, ayuda a detectar a tiempo las posibles acciones de usuarios maliciosos. Dentro de las aplicaciones que existen sobre este tema, son los análisis de logs de los sistemas, con cuya información se puede verificar si existe algún \textit{bug}, error o anomalía, además de ver si existe algún intruso o violación al sistema.

Entre los sistemas actuales de procesamiento de \textsl{stream} se encuentran S4 \citep{s4yahoo}, Storm \citep{stormtwitter}, Samza \citep{samza}, entre otros, los cuales son los más utilizados como arquitectura de procesamiento en la confección de distintas aplicaciones de \textsl{stream}. Este tipo de arquitecturas está basada en grafos, donde las vértices son operaciones realizadas al flujo de datos que es enviado por las distintas aristas. Por lo que el usuario diseña la topología a su conveniencia, según la necesidad que posea el sistema, creando así una aplicación. Aunque poseen bastante flexibilidad para la creación de diversas aplicaciones, por la facilidad de crear distintas topologías, no lo tiene para adaptarse en el tiempo a las condiciones del tráfico entrantes, esto debido a que las topolog\'ias de procesamiento generadas son est\'aticas. Dada la naturaleza din\'amica de las interacciones, pueden surgir problemas de sobrecarga en algunas partes de la topología asociada a la aplicación.

El problema de sobrecarga conlleva a una baja en el rendimiento, produciendo una pérdida de recursos, tiempo e información. Abordar este problema es crítico, puesto que al realizar una optimización en el sistema, implica una disminución en el tiempo de procesamiento, por lo que hay una mayor cantidad de datos procesados en un período de tiempo, lo cual conlleva a una mayor precisión en los resultados obtenidos de la aplicación.

Lo anterior lo podemos entender de mejor manera con el siguiente ejemplo: se posee un tiempo $t$ para procesar $n$ datos, de disminuir el tiempo de procesamiento total de los datos, se tiene que en el mismo tiempo $t$ se procesarán una cantidad $n+m$ de datos, donde $m$ son los datos adicionales a analizar debido a la mejora del rendimiento. Como existe un aumento en la cantidad de datos procesados, la información procesada posee mayor precisión, debido que se poseen mayor cantidad de datos con los que analizar el comportamiento estudiando. Por ejemplo, al procesar una mayor cantidad de transacciones en la bolsa de comercio, se puede poseer una predicción más precisa de cómo se comportará la bolsa a futuro. Desde otro punto de vista, se efectúa una mejora en los recursos utilizados, habiendo una disminución de recursos ociosos que genera la sobrecarga en el operador.

\section{Descripción del problema}
\label{intro:problema}

%En los SPS (Sistemas de Procesamiento de \textit{Streaming}) son planteados como un grafo en el cual cada vértice es un operador y las aristas son el flujo de datos entre operadores. Debido a esto, puede generarse alguna sobrecarga del sistema, dado distintos factores, entre los cuales puede ser físico o lógico. El físico se define como los componentes que posea la máquina que generan una limitante en el sistema. En cambio, el lógico es una limitante dado los componentes del grafo generador por el sistema.

Los SPS (Sistemas de Procesamiento de \textit{Streaming}) modelan sus aplicaciones como un grafo cuyas vértices son operadores y las aristas son flujo de datos entre los operadores. Dada el carácter estático de su representación en forma de grafo, puede existir sobrecarga del sistema a producto de factores físicos o lógicos. El factor físico se define como los componentes que posee la máquina, los cuales pueden ser limitantes para el sistema alojado. En cambio, el lógico se concentra en los componentes del grafo, por lo que existe una limitante en la cantidad de operadores o la cantidad de flujo existente entre los operadores.

%Es por esto, que el problema se plantea como la sobrecarga que puede existir en el sistema debido a los distintos factores lógicos, como la cola de cada operador, siendo esto a causa de la falta de flexibilidad del SPS, en los operadores más demandados. Esto debido a que no existe una forma de disminuir la sobrecarga y reducir las colas de espera, para mejorar el rendimiento del sistema y obtener información cercana al tiempo real.

La falta de flexibilidad del SPS en los operadores más demandados. Esto sucede dada la condición estática del grafo, es decir la topología del grafo no cambia con el tiempo, por lo que no existe una forma para adaptarse el tráfico de manera dinámica que permita variar la carga y reducir las colas, de tal manera de mejorar el rendimiento del sistema y obtener información más precisa y en tiempo real.

\section{Solución propuesta}
\label{intro:solucion}

La solución propuesta consiste en el dise\~no de algoritmos de predicci\'on y distribuci\'on de carga a nivel de la l\'ogica del grafo, los cuales adaptan el grafo a las variaciones del grafo. Por lo que se propone implementar cuatro módulos que componen la estructura del sistema de distribución de carga: monitor de carga, analizador de carga, predictor de carga y administrador de réplicas.

El monitor de carga está encargado de recuperar el nivel de carga de cada uno de los operadores. Esta información es entregada a los módulos de analizador y predictor de carga, los cuales están encargados de evaluar si existe sobrecarga en el operador. Cada uno de éstos trabaja de forma independiente y tiene distintos métodos, uno proactivo y otro reactivo, de tal manera de poseer mayor exactitud en la detección de una sobrecarga.

El analizador de carga consiste en un método reactivo, el cual analiza el tráfico de los operadores en el tiempo actual, y cuantifica su carga. El estado de la carga de cada operador depende de un umbral, por lo que según ésto se envía al administrador de réplica el tráfico de cierto operador de ser necesario una replicación.

El predictor de carga consiste en un método proactivo, el cual analiza la carga de los distintos operadores según una ventana de tiempo, y predice la carga para la siguiente ventana de tiempo. De esta manera, se determina la posible carga que existe en cierto período de tiempo futuro, información que utiliza el administrador de réplicas para determinar la mejor configuración de los operadores para dicho período.

El administrador de réplicas se alimenta de la información entregada por los dos módulos anteriores, y así toma una decisión de la administración de cada una de las réplicas de los distintos operadores. Por lo tanto, verifica cuántas réplicas son necesarias según la cantidad de tráfico de cierto operador.

Finalmente, el sistema de procesamiento constantemente está realizando un \textit{feedback} al sistema de optimización, de tal manera que pueda administrar las réplicas necesarias. De esta manera, se dispone de un sistema capaz de procesar información de manera más rápida, a través de este sistema de optimización con bajo \textit{overhead}.


\section{Objetivos y alcance del proyecto}
\label{intro:objetivos}

\subsection{Objetivo general}
	Dise\~no, construcción y evaluaci\'on de un algoritmo de predicci\'on y un algoritmo de distribuci\'on de carga para sistemas de procesamiento de \textit{stream}.

\subsection{Objetivos específicos}
\begin{enumerate}
	\item Dise\~nar e implementar un algoritmo reactivo que permita analizar en el momento la carga de los operadores.
	\item Dise\~nar e implementar un algoritmo de predicci\'on que permita estimar la carga de los operadores.
	\item Dise\~nar e implementar un algoritmo de distribuci\'on que permita la administraci\'on de los operadores del grafo de procesamiento de forma el\'astica.
	\item Dise\~nar y construir experimentos que permitan validar la hip\'otesis formulada.
	\item Evaluar y analizar el rendimiento del sistema a trav\'es de aplicaciones generadas sobre sistemas de procesamiento de \textit{stream}.
\end{enumerate}

\subsection{Alcances}
Dentro de los alcances y limitaciones que se tienen en el proyecto son:
\begin{itemize}
	\item La evaluación de la solución presentada se implementará sobre un solo sistema de procesamiento de \textit{stream}.
	\item Los datos emitidos de la fuente de datos son homogéneos, teniendo una tasa de servicio similar.
	\item La distribución de flujo de datos es a nivel de operadores y no de nodos f\'isicos, por lo que no se analizó la carga de estos \'ultimos.
	\item Los algoritmos propuestos no incluyen t\'ecnicas que garanticen el procesamiento de todo el flujo de datos.
	\item En la evaluación de los algoritmos propuestos se consideró el costo de comunicación de manera igualitaria para todos los operadores.
	%\item Se comparará la solución con dos motores de procesamiento de \textit{stream} del estado del arte.
\end{itemize}


\section{Metodología y herramientas utilizadas}
\label{intro:metodologia}

\subsection{Metodología}
Dado el carácter de investigación de la propuesta de tesis, se propone utilizar el método científico para la realización de ésta. Dentro de las etapas propuesta por \citep{hernandez2010metodologia} están:

\begin{enumerate}
	\item Formulación de la hipótesis: ``La utilización de un sistema de distribución de carga, el cual contenga un modelo reactivo y predictivo, de tal manera que permita mejorar la distribución de carga entre los operadores de manera dinámica, logrando aumentar el rendimiento del SPS, de tal manera que aumente la cantidad de eventos procesados".
	\item Elaboración del marco teórico: Exponer las investigaciones que existen sobre problemas de sobrecarga en los operadores de SPS. Así mismo, los conceptos fundamentales de estos sistemas.
	\item Seleccionar el diseño apropiado de investigación: Diseñar el experimento para el problema de balance de carga a nivel lógico en un SPS, vale decir, los algoritmos de predicción y distribución. Cada ejecución de los experimentos se basan según los principios de un SPS.
	\item Analizar los resultados: De deberá analizar los resultados según las estadísticas entregadas y el modelo propuesto.
	\item Presentar los resultados: Elaborar el reporte de investigación y presentar los resultados en gráficos y tablas.
	\item Concluir en base a los resultados de la investigación.
\end{enumerate}

\subsection{Herramientas de desarrollo}
Para el procesamiento de \textit{stream} se utilizó Apache S4 0.6.0, por lo que fue necesario para su configuración Java SE Development Kit 7. Dentro esto, el lenguaje de programación de cada una de las estructuras del sistema desarrollado fue en Java, por lo que se trabajó sobre el IDE Eclipse Standard 4.4.2, y para el prototipo del modelo matemático se utilizó MATLAB 2014a. De forma complementaria, se utilizó Texmaker 4.1 para la confección de los distintos informes requeridos y la documentación correspondiente al trabajo.

\section{Organización del documento}
\label{intro:organizacion}
En el presente documento se divide en seis capítulos. En el primer capítulo se presenta la problemática y la solución propuesta, conjunto con los objetivos y la metodología utilizada. En el segundo capítulo se exponen los conceptos teóricos involucrados. Posteriormente, el tercer capítulo aborda los distintos enfoques y técnicas que se han brindado en la literatura para dar soluciones al problema planteado. Luego, el cuarto capítulo se describen el diseño de los algoritmos utilizados en el sistema propuesto, explicando las distintas decisiones que se tomaron para el diseño de éste. En el quinto capítulo se presentan los distintos experimentos realizados para evaluar el sistema diseñado, donde se explica su implementación y evaluación según los experimentos diseñados. Finalmente, el sexto capítulo se exponen las respectivas conclusiones obtenidas a partir del presente trabajo.