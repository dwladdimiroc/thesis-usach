\chapter{Introducci\'on}
\label{cap:introduccion}

\section{Antecedentes y motivaci\'on}
\label{intro:motivacion}

La gran contribuci\'on de informaci\'on en la Internet se ha debido al origen de la Web 2.0 donde \'esta se caracteriza por la participaci\'on activa del usuario, \normalsize{como} en blogs, redes sociales u otras aplicaciones Web \citep{web2007oberhelman}.

%Con el objetivo de extraer informaci\'on de dichos datos, se crean sistemas de procesamiento para grandes cantidades de informaci\'on generadas por la interacci\'on entre los usuarios.

Con el paso del tiempo, m\'as y m\'as informaci\'on es generada por distintas interacciones generadas por los usuarios. \normalsize{Por lo que analizar} o extraer esta informaci\'on no es una tarea f\'acil, m\'as a\'un cuando muchas de estas interacciones deben ser analizadas en tiempo real, dada su dependencia temporal. Por esta \'ultima caracter\'istica es que sistemas tradicionales de procesamiento basados en MapReduce \citep{2010Lin} o \textit{bash processing} \citep{HawwashN14} no son ideales para el an\'alisis de esta informaci\'on.

Es as\'i como con el tiempo se han ido creando distintos sistemas de procesamiento capaces de lidiar con las restricciones de temporalidad, debido al interesante funcionamiento que poseen, las que se caracterizan por ser capaces de procesar grandes flujos de datos en tiempo real \citep{ChenZ14a}. El requisito de procesar informaci\'on en tiempo real surge por la necesidad de los usuarios en obtener respuestas r\'apidas y actualizadas que le permitan tomar decisiones en per\'iodos cortos de tiempo. Dentro de los ejemplos existentes se encuentran: an\'alisis de sentimientos de los mensajes de usuarios, an\'alisis de los precios de la bolsa de valores, recopilaci\'on de informaci\'on en caso de emergencia, entre otros. Este tipo de aplicaciones son necesarias para sus usuarios, debido que proveen de informaci\'on actualizada que permite mejorar entre otros casos la toma de decisiones \citep{Wenzel14}.

Un ejemplo de esto son las aplicaciones que analizan redes sociales en caso de un desastre natural, donde grandes cantidades de informaci\'on son generadas, y se requiere procesar esta informaci\'on lo m\'as cercano al tiempo real para obtener informaci\'on que permita un trabajo de recuperaci\'on m\'as eficiente dado el suceso \citep{andrade2014fundamentals}. De esta manera, se puede construir un sistema que procese los datos realizando un an\'alisis de la percepci\'on de la gente, b\'usqueda de personas desaparecidas o focos de necesidad. Con esta informaci\'on, se puede establecer sectores cr\'iticos, facilitar la b\'usqueda de personas, distribuci\'on de alertas, o detecci\'on de necesidades, lo cual es crucial para tomar decisiones en esos momentos.

Por otra parte, tambi\'en son utilizados para llevar a cabo predicciones en la bolsa de comercio\normalsize{. De esta manera,} se crean sistemas de procesamiento que apliquen modelos matem\'aticos, los cuales permiten predecir el comportamiento para el siguiente d\'ia en el mercado. Con estos sistemas, la ganancia que existe por parte de las personas interesadas puede aumentar considerablemente, por lo que ha generando un alto inter\'es en el desarrollo e investigaci\'on en esta \'area.

Tambi\'en se aplica en casos de seguridad en redes, donde se permite realizar un monitoreo de las actividades ocurridas en la red. Como la informaci\'on es procesada en tiempo real, ayuda a detectar a tiempo las posibles acciones de usuarios maliciosos. Dentro de las aplicaciones que existen sobre este tema, \normalsize{son el an\'alisis de los logs de la red cerrada}, donde con esta informaci\'on se puede verificar si existe alg\'un \textit{bug}, error o anomal\'ia, adem\'as de ver si existe alg\'un intruso o violaci\'on al sistema.

Para dar soporte a estas aplicaciones existen los SPS (Sistemas de Procesamiento de \textit{Stream}) tales como S4 \citep{s4yahoo}, Storm \citep{stormtwitter}, Samza \citep{samza}, entre otros. El paradigma de procesamiento de estos sistemas se basa en grafos, donde los v\'ertices son operaciones realizadas al flujo de datos, representadas por las distintas aristas. Para la generaci\'on de una aplicaci\'on, el usuario debe dise\~nar una topolog\'ia de procesamiento compuesto por las tareas u operaciones deseadas. Cada grafo o topolog\'ia tiene un \textit{input} desde una fuente de datos, y un \textit{output} del flujo de salida proveniente del \'ultimo operador. Aunque poseen bastante flexibilidad para la creaci\'on de diversas aplicaciones, por la facilidad de crear distintas topolog\'ias, no lo tiene para adaptarse con el tiempo a las condiciones del tr\'afico \normalsize{entrante} cuando se encuentran en funcionamiento, esto debido a que las topolog\'ias de procesamiento generadas son est\'aticas en tiempo de ejecuci\'on. Dada la naturaleza din\'amica de las interacciones, pueden surgir problemas de distribuci\'on de carga en la topolog\'ia asociada a la aplicaci\'on.

El problema de sobrecarga conlleva a una baja en el rendimiento, produciendo una p\'erdida de recursos, tiempo e informaci\'on. Abordar este problema es cr\'itico, puesto que al realizar una optimizaci\'on en el sistema, implica un aumento en la cantidad de datos procesados, y una mayor precisi\'on en los resultados por parte de la aplicaci\'on.

Lo anterior lo podemos entender de mejor manera con el siguiente ejemplo: se posee un tiempo $t$ para procesar $n$ datos\normalsize{. Si se aumenta} la cantidad de datos procesados, se tiene que en el mismo tiempo $t$ se procesan una cantidad $n+m$ de datos, donde $m$ son los datos adicionales a analizar debido a la mejora del rendimiento. Como existe un aumento en la cantidad de datos procesados, la informaci\'on obtenida puede ser m\'as precisa, dado que se posee una mayor cantidad de datos. Por ejemplo, al procesar una mayor cantidad de transacciones en la bolsa de comercio, se puede poseer una predicci\'on m\'as precisa de c\'omo se comporta la bolsa a futuro. Desde otro punto de vista, se efect\'ua una mejora en los recursos utilizados, habiendo una disminuci\'on de los recursos ociosos.

\section{Descripci\'on del problema}
\label{intro:problema}

%Dada la rigidez de la topolog\'ia, la cual no var\'ia en el transcurso del procesamiento, y el car\'acter altamente din\'amico del tr\'afico, y los posibles \textit{peaks} que pueden alteran el tr\'afico entrante, pueden generarse problemas de balance de carga entre los operadores de la topolog\'ia, sobrecargando alguno de estos y comprometiendo el rendimiento del sistema.

Dado el car\'acter est\'atico del grafo de procesamiento en tiempo de ejecuci\'on y el car\'acter altamente din\'amico del tr\'afico, puede surgir problemas de balance de carga entre los operadores de la topolog\'ia, sobrecargando alguno de estos y comprometiendo el rendimiento del sistema.

\section{Soluci\'on propuesta}
\label{intro:solucion}

La soluci\'on propuesta consiste en un modelo el\'astico de replicaci\'on de operadores para los sistemas de procesamiento de \textit{stream}, el cual permite adaptar el grafo de procesamiento a las variaciones del tr\'afico. Para esto se ha implementado cuatro m\'odulos que componen la estructura del modelo el\'astico: monitor de carga, analizador de carga, predictor de carga y administrador de r\'eplicas.

El monitor de carga est\'a encargado de recuperar el nivel de carga de cada uno de los operadores. Esta informaci\'on es entregada a los m\'odulos analizador y predictor de carga, los cuales est\'an encargados de medir el nivel de carga del operador, y seg\'un eso modificar la cantidad de r\'eplicas necesarias. Cada uno de estos m\'odulos trabaja de forma independiente y poseen distintos enfoques: reactivo y predictivo.

El analizador de carga aplica un enfoque reactivo, el cual analiza el tr\'afico de los operadores en el tiempo actual y cuantifica su carga. El estado de la carga de cada operador depende de un umbral, por lo que seg\'un \'este se solicita al administrador de r\'eplica incrementar o disminuir la cantidad de r\'eplicas del operador.

El predictor de carga aplica un enfoque predictivo, el cual analiza la carga de los distintos operadores en una ventana de tiempo y predice la carga para la siguiente ventana. Con esta informaci\'on el administrador de r\'eplicas determina la mejor configuraci\'on de los operadores para dicho per\'iodo.

El administrador de r\'eplicas por su parte se alimenta de la informaci\'on entregada por los dos m\'odulos anteriores, y en base a esto, toma una decisi\'on respecto a los recursos asignados al operador. En otras palabras, verifica cu\'antas r\'eplicas son necesarias seg\'un el tama\~no del tr\'afico.

\section{Objetivos y alcance del proyecto}
\label{intro:objetivos}

\subsection{Objetivo general}
	Dise\~no, construcci\'on y evaluaci\'on de un modelo el\'astico de replicaci\'on de operadores para un sistema de procesamiento de \textit{stream} en tiempo real.

\subsection{Objetivos espec\'ificos}
\begin{enumerate}
	\item Dise\~nar e implementar un algoritmo reactivo que permita analizar en el momento la carga de los operadores.
	\item Dise\~nar e implementar un algoritmo de predicci\'on que permita estimar la carga de los operadores.
	\item Dise\~nar e implementar un algoritmo que permita la administraci\'on del n\'umero de operadores del grafo de procesamiento de forma el\'astica.
	\item Dise\~nar y construir experimentos que permitan validar la hip\'otesis formulada.
	\item Evaluar y analizar el rendimiento del modelo a trav\'es de aplicaciones generadas sobre un sistema de procesamiento de \textit{stream}.
\end{enumerate}

\subsection{Alcances}
Dentro de los alcances y limitaciones que se tiene en el proyecto son:
\begin{enumerate}
	\item La evaluaci\'on de la soluci\'on se ha implementado sobre un solo sistema de procesamiento de \textit{stream}.
	\item Los datos emitidos de la fuente de datos son homog\'eneos, teniendo una tasa de procesamiento \normalsize{constante para cada operador.}
	\item La distribuci\'on de flujo de datos es a nivel de operadores y no de m\'aquinas, por lo que no se ha analizado la carga de estos \'ultimos.
	\item Los algoritmos propuestos no incluyen t\'ecnicas que garanticen el procesamiento de todo el flujo de datos.
	\item En la evaluaci\'on de los algoritmos propuestos se ha considerado el costo de comunicaci\'on de manera igualitaria para todos los operadores.
\end{enumerate}


\section{Metodolog\'ia y herramientas utilizadas}
\label{intro:metodologia}

\subsection{Metodolog\'ia}
Dado el car\'acter de investigaci\'on del trabajo propuesto, se ha utilizado el m\'etodo cient\'ifico para la realizaci\'on de \'este. Dentro de las etapas propuesta por \citep{hernandez2010metodologia} est\'an:

\begin{enumerate}
	\item Formulaci\'on de la hip\'otesis: ``La utilizaci\'on de un modelo el\'astico en un sistema de procesamiento de \textit{stream} permitir\'a aumentar la cantidad de eventos procesados, y con ello la precisi\'on de los resultados".
	\item Elaboraci\'on del marco te\'orico: Exponer los trabajos que existen sobre problemas de carga en los operadores de SPS. As\'i mismo, los conceptos fundamentales de estos sistemas.
	\item Seleccionar el dise\~no apropiado de investigaci\'on: Dise\~nar el modelo el\'astico para el problema de balance de carga a nivel l\'ogico en un SPS, vale decir, los distintos m\'odulos, donde se posea un algoritmo reactivo y otro predictivo. Cada dise\~no y ejecuci\'on de los experimentos se basan en los principios de un SPS definiendo m\'etricas acordes para dicho modelo de procesamiento.
	\item Analizar los resultados: De debe analizar los resultados seg\'un las m\'etricas establecidas y el modelo propuesto.
	\item Presentar los resultados: Elaborar el reporte de investigaci\'on y presentar los resultados en gr\'aficos y tablas.
	\item Concluir en base a los resultados de la investigaci\'on.
\end{enumerate}

\subsection{Herramientas de desarrollo}
Para el procesamiento de \textit{stream} se ha utilizado Apache S4 0.6.0, por lo que es necesario para su configuraci\'on Java SE Development Kit 7. Dentro esto, el lenguaje de programaci\'on de cada uno de los m\'odulos del modelo desarrollado se ha utilizado el lenguaje de programaci\'on Java, y se ha trabajado sobre el IDE Eclipse Standard 4.4.2. De forma complementaria, se ha utilizado Texmaker 4.1 para la confecci\'on de los distintos informes requeridos y la documentaci\'on correspondiente al trabajo.

\section{Organizaci\'on del documento}
\label{intro:organizacion}
En el presente documento se divide en seis cap\'itulos. En \normalsize{este} cap\'itulo se presenta la problem\'atica y la soluci\'on propuesta, en conjunto con los objetivos y la metodolog\'ia utilizada. En el segundo cap\'itulo se exponen los conceptos te\'oricos involucrados. Posteriormente, el tercer cap\'itulo aborda los distintos enfoques y t\'ecnicas que se han brindado en la literatura para dar soluciones al problema planteado. Luego, el cuarto cap\'itulo se describe el dise\~no de los algoritmos utilizados en el modelo propuesto, explicando las distintas decisiones que se han tomado para el dise\~no de \'este. En el quinto cap\'itulo se presentan los distintos experimentos realizados para evaluar el sistema dise\~nado, donde se explica su implementaci\'on y evaluaci\'on seg\'un los experimentos dise\~nados. Finalmente, en el sexto cap\'itulo se exponen las respectivas conclusiones obtenidas a partir del presente trabajo.