\chapter{Modificaciones al c\'odigo fuente de S4}
\label{apendice:codigoFuenteS4}

En el C\'odigo \ref{lst:monitor} se presenta la implementanci\'on de las tareas que est\'an a cargo del env\'io de estad\'isticas al sistema de distribuci\'on de carga, donde una de ellas est\'a a cargo de obtener las muestras para el historia, y la otra est\'a a cargo de enviar las estad\'isticas de los PE existentes en el sistema. Adem\'as de esto, para la ejecuci\'on del sistema, se debe esperar que el \textit{Adapter} est\'e ejecut\'andose, por lo que espera la notificaci\'on por parte de \'este para la ejecuci\'on de las tareas.

\begin{lstlisting}[caption={Tareas que ejecutan el sistema de distribuci\'on de carga.},label={lst:monitor},language=Java]
private void startMonitor() {
	if (runMonitor) {
		synchronized (getBlockAdapter()) {
			try {
				getBlockAdapter().wait();
			} catch (InterruptedException e) {
				getLogger().error(e.getMessage());
			}

			ScheduledExecutorService getEventCount = Executors
					.newSingleThreadScheduledExecutor();
			getEventCount.scheduleAtFixedRate(new OnTimeGetEventCount(),
					1000, 1000, TimeUnit.MILLISECONDS);

			ScheduledExecutorService sendStatus = Executors
					.newSingleThreadScheduledExecutor();
			sendStatus.scheduleAtFixedRate(new OnTimeSendStatus(), 6000,
					5000, TimeUnit.MILLISECONDS);
		}
	} else {
		ScheduledExecutorService sendStatus = Executors
				.newSingleThreadScheduledExecutor();
		sendStatus.scheduleAtFixedRate(new OnTimeSendStatus(), 6000, 5000,
				TimeUnit.MILLISECONDS);
	}

}
\end{lstlisting}

En el C\'odigo \ref{lst:add} se muestra la implementanci\'on que realizada para a\~nadir una r\'eplica a un PE en espec\'ifico. El tipo \textit{StatusPE} hace referencia un objeto creado en la implementanci\'on, para almacenar los datos y estad\'isticas correspondientes al PE en el an\'alisis de carga seg\'un el sistema de distribuci\'on de carga, como la cantidad de r\'eplicas deseadas.

\begin{lstlisting}[caption={A\~nadir r\'eplicas a un PE en S4.},label={lst:add},language=Java]
public void addReplication(StatusPE statusPE) {
	for (Streamable<Event> stream : getStreams()) {
		for (ProcessingElement PEPrototype : stream.getTargetPEs()) {
			if (PEPrototype.getClass().equals(statusPE.getPE())) {
				for (long i = PEPrototype.getNumPEInstances(); i < statusPE
						.getReplication(); i++) {
					PEPrototype.getInstanceForKey(Long.toString(i));
				}
			}
		}
	}
}
\end{lstlisting}

En el C\'odigo \ref{lst:remove} se muestra la implementanci\'on que realizada para eliminar una r\'eplica a un PE en espec\'ifico.

\begin{lstlisting}[caption={Eliminar r\'eplicas a un PE en S4.},label={lst:remove},language=Java]
public void removeReplication(StatusPE statusPE) {
	for (Streamable<Event> stream : getStreams()) {
		for (ProcessingElement PEPrototype : stream.getTargetPEs()) {
			if (statusPE.getPE().equals(PEPrototype.getClass())) {
				for (int i = statusPE.getReplication(); i < PEPrototype
						.getInstances().size(); i++) {
					ProcessingElement peCurrent = PEPrototype
							.getInstanceForKey(Integer.toString(i));
					peCurrent.close();
				}
			}
		}
	}
}
\end{lstlisting}