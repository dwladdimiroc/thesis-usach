\chapter{Configuraci\'on para la comunicaci\'on de S4}
\label{apendice:config-comm-S4}

En la tabla \ref{tab:config-comm-s4} se muestra los par\'ametros utilizados para la configuraci\'on para la comunicaci\'on de S4. La descripci\'on de cada uno de los par\'ametros est\'a en el proyecto de S4 en la carpeta de comunicaci\'on.

\begin{table}[!ht]
\centering
<<<<<<< HEAD
\captionsetup{justification=centering}
\caption[Parámetros de la configuración para la comunicación de S4.]{Parámetros de la configuración para la comunicación de S4.\\Fuente: Elaboración propia.}
=======
\caption{Par\'ametros de la configuraci\'on para la comunicaci\'on de S4.}
>>>>>>> 2593dcc9f1f0f407aaed2b40f87140ab2e858079
\begin{tabular}{|l|l|}
\hline
Par\'ametro & Valor \\ \hline
s4.comm.emitter.class & org.apache.s4.comm.tcp.TCPEmitter \\
s4.comm.emitter.remote.class & org.apache.s4.comm.tcp.TCPRemoteEmitter \\
s4.comm.listener.class & org.apache.s4.comm.tcp.TCPListener \\
s4.comm.timeout & 1000 \\
s4.sender.parallelism & 5 \\
s4.sender.workQueueSize & 10000 \\
s4.sender.maxRate & 10000 \\
s4.remoteSender.parallelism & 5 \\
s4.remoteSender.workQueueSize & 100000 \\
s4.remoteSender.maxRate & 10000 \\
s4.emitter.maxPendingWrites & 1000 \\
s4.stream.workQueueSize & 1000000 \\ \hline
\end{tabular}
\label{tab:config-comm-s4}
\end{table}