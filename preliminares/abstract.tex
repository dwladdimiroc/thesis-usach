\resumenCastellano{
En el mundo actual de la información, grandes cantidades de datos son generados cada segundo desde las más diversas fuentes: redes sociales, redes de sensores, buscadores Web, entre otros. Extraer información de dichos datos muchas veces requiere que este análisis sea llevado a cabo en tiempo real, debido que el análisis que se deben realizar depende de la temporalidad. Para lograr procesar estas grandes cantidades de datos, existen sistemas especializados llamado sistemas de procesamiento \textit{stream} (SPS), los cuales pueden procesar en tiempo real los datos que van llegando por una o más fuentes de datos. Estos sistemas están basados en grafos, cuyos vértices realizan operaciones según el flujo de dato que van llegando por las aristas del grafo. Debido a esto, existe un problema con la topología del grafo de la aplicación, dado que ésta al iniciar el sistema posee una determinada forma, la cual puede ser que no se adapte a la cantidad de flujo de datos entrante, pudiendo generar sobrecargas en un operador. Dado esto, se planteó un sistema de distribución de carga que optimice el rendimiento de cada operador. Para esto se diseñó un algoritmo reactivo, usando la técnica de fisión, y otro predictivo, usando cadena de Márkov, para poder detectar sobrecargas en un operador, de tal manera de indicarlo y optimizar el rendimiento del sistema. Los resultados obtenidos de los experimentos realizados en el SPS S4, se encuentra una mejora de hasta de tres veces más en el procesamiento de los eventos, con un costo asociado a un aumento de $0,0119\%$ del uso de la CPU, pero una disminución de un $1,5187\%$ en el consumo de memoria RAM con el experimento realizado. De esta manera, se cumplió con un sistema que optimizara el rendimiento de un SPS, con un bajo \textit{overhead} de implementación.
\vspace*{0.5cm}
\KeywordsES{SPS; Elasticidad; Distribución de carga; Balance de carga; Algoritmos reactivos; Fisión; Algoritmos predictivos; Cadena de Márkov}
}

\newpage

\resumenIngles{
In the actual world of information, great quantities of data are generated every second from the most diverse sources: social networks, sensor networks, web searchers, among others. Extract information from that data requires a real-time analysis, because the analysis depends of the temporality. To process this big quantity of data, there are specialized systems called stream processing streams, which can process data from diverse sources in real time. This systems are based on graphs, which vertices operate depending of the incoming stream of data through their edges. Because of this, exists a big problem with the application graph topology, because when the system starts it has a shape that cannot be changeable and adaptable to the incoming streams of data generating overhead on one operator. Because of this, a distribution load system has been proposed to optimize the performance of each operator. For two algorithm has been designed, reactive algorithm using the fision technique, and a predictive algorithm using Markov chains to detect overheads on an operator and to indicate and optimize the system performance. The obtained results from the test in S4 show an improvement of 3 times more on the events processing, with an asociated cost of a 0.0119% more of CPU utilization, but with a decrease of 1.5187% of the RAM consumption. This way, a system that can optimize the performance of an SPS has been fulfilled with a low overhead implementation.
\vspace*{0.5cm}
\KeywordsEN{SPS; Elastic; Load balancing; Reactive Algorithm; Fision; Predictive Algorithm; Markov chain}
}
