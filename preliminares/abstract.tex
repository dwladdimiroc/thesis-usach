\resumenCastellano{
En el mundo actual de la información, grandes cantidades de datos son generados cada segundo desde las más diversas fuentes: redes sociales, redes de sensores, buscadores Web, entre otros. Extraer información de dichos datos muchas veces requiere que este procesamiento sea llevado a cabo en tiempo real, debido a que el análisis que se debe realizar depende de la temporalidad en que son generadas los eventos. Para lograr procesar grandes cantidades de datos con estas restricciones, existen sistemas especializados llamado sistemas de procesamiento de \textit{stream} (SPS), los cuales pueden procesar en tiempo real los datos que van llegando por una o más fuentes de datos. Estos sistemas están basados en grafos, cuyos vértices realizan operaciones sobre un flujo de datos que va llegando reflejada por las aristas del grafo. La topología del grafo le brinda flexibilidad al SPS para generar diversas aplicaciones de procesamiento, sin embargo, dicha topología es estática una vez el sistema se ejecuta. Dado este problema, este trabajo se plantea un modelo elástico que sea capaz de adaptar la topología del grafo a las condiciones del tráfico existente. Para esto se ha diseñado un algoritmo reactivo, usando la técnica de fisión, y otro predictivo, usando cadena de Márkov, ambas técnicas permiten estimar la carga de los operadores, y adaptar el grafo acorde a lo indicado por estos. Dicha modificación consiste en incrementar o disminuir la cantidad de réplicas de un operador según su nivel de carga. Los resultados obtenidos de los experimentos realizados en el SPS S4, muestran una mejora de hasta nueve veces más eventos procesados, con un costo asociado a un aumento de $0,01\%$ del uso de la CPU, pero una disminución de un $1,5\%$ en el consumo de memoria RAM.
\vspace*{0.5cm}
\KeywordsES{SPS; Elasticidad; Distribución de carga; Balance de carga; Algoritmos reactivos; Fisión; Algoritmos predictivos; Cadena de Márkov}
}

\newpage

\resumenIngles{
In the actual world of information, great quantities of data are generated every second from the most diverse sources: social networks, sensor networks, web searchers, among others. Extract information from that data requires a real-time process to be done due to the analysis that depends of the temporality in which the events are generated. To process this big quantity of data with this constraints, there are specialized systems called stream processing streams, which can process data from diverse sources in real time the data coming from one or more sources. This systems are based on graphs, which vertices operate depending of the incoming stream of data through their edges. This topology gives a certain flexibility to the SPS to generate diverse processing applications. However, this topology is static at the time that is executed. Given this problem, this work proposes an elastic model that is able to adapt the graph topology to the existing traffic conditions. A reactive algorithm have been designed for this, using the fission technique and a preactive one, using Markov chain, this two techniques allows to estimate de operators loads and to adapt the graph according of what they are indicating. This modification consists on increase and decrease the quantity of operator replicas depending of the load level. The obtained results from the experiments done in the SPS S4, show an improvement up to nine times more processed events, with an associated cost of $0.01\%$ if CPU use, but with a decrease of $1.5\%$ RAM consumption.
\vspace*{0.5cm}
\KeywordsEN{SPS; Elastic; Load balancing; Reactive Algorithm; Fision; Predictive Algorithm; Markov chain}
}
