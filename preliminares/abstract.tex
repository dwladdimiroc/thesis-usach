\resumenCastellano{
En el mundo actual de la información, grandes cantidades de datos son generados cada segundo desde las más diversas fuentes: redes sociales, redes de sensores, buscadores Web, entre otros. Extraer información de dichos datos muchas veces requiere que este análisis sea llevado a cabo en tiempo real, debido que el análisis que se deben realizar depende de la temporalidad. Para lograr procesar estas grandes cantidades de datos, existen sistemas especializados llamado sistemas de procesamiento \textit{stream} (SPS), los cuales pueden procesar en tiempo real los datos que van llegando por una o más fuentes de datos. Estos sistemas están basados en grafos, cuyos vértices realizan operaciones según el flujo de dato que van llegando por las aristas del grafo. Debido a esto, existe un problema con la topología del grafo de la aplicación, dado que ésta al iniciar el sistema posee una determinada forma, la cual puede ser que no se adapte a la cantidad de flujo de datos entrante, pudiendo generar sobrecargas en un operador. Dado esto, se planteó un sistema de distribución de carga que optimice el rendimiento de cada operador. Para esto se diseñó un algoritmo reactivo, usando la técnica de fisión, y otro predictivo, usando cadena de Márkov, para poder detectar sobrecargas en un operador, de tal manera de indicarlo y optimizar el rendimiento del sistema. Los resultados obtenidos de los experimentos realizados en el SPS S4, se encuentra una mejora de hasta de tres veces más en el procesamiento de los eventos, con un costo asociado a un aumento de $0,0119\%$ del uso de la CPU, pero una disminución de un $1,5187\%$ en el consumo de memoria RAM con el experimento realizado. De esta manera, se cumplió con un sistema que optimizara el rendimiento de un SPS, con un bajo \textit{overhead} de implementación.
\vspace*{0.5cm}
\KeywordsES{SPS; Elasticidad; Distribución de carga; Balance de carga; Algoritmos reactivos; Fisión; Algoritmos predictivos; Cadena de Márkov}
}

\newpage

\resumenIngles{
In today's world of information, large amounts of data are generated every second from the most diverse sources: social networks, sensor networks, Web search engines, among others. Extract information from the data often requires that this analysis is carried out in real time, because the analysis to be performed depends on the timing. To achieve process these large amounts of data, there are specialized systems called stream processing systems (SPS), which can process data in real time as they arrive for one or more data sources. These systems are based on graphs whose vertices perform operations according to the flow of data that arrive from the edges of the graph. Because of this, there is a problem with the topology graph of the application, since it at system startup has a certain way, which might not suit the amount of incoming data stream, potentially leading to overcharge operator. Given this, a charge distribution system that optimizes the performance of each operator was raised. For this a reactive algorithm was designed, using the technique of fission, and other predictive, using Markov chain, to detect overloading an operator, so to indicate and optimize system performance. The results of experiments on the SPS S4, is improved up to three times in the processing of events at a cost associated with an increase of $0.0119\%$ of CPU usage, but a decrease a $1.5187\%$ consumption RAM with experiment. Thus it was fulfilled with a system that optimizes the performance of an SPS, with low overhead implementation.
\vspace*{0.5cm}
\KeywordsEN{SPS; Elastic; Load balancing; Algorithm reactive; Fision; Algorithm predictive; Markov chain}
}
