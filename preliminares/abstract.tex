\resumenCastellano{
En el mundo actual de la informaci\'on, grandes cantidades de datos son generados cada segundo desde las m\'as diversas fuentes: redes sociales, redes de sensores, buscadores Web, entre otros. Extraer informaci\'on de dichos datos muchas veces requiere que este procesamiento sea llevado a cabo en tiempo real, debido a que el an\'alisis que se debe realizar depende de la temporalidad en que son generadas los eventos. Para lograr procesar grandes cantidades de datos con estas restricciones, existen sistemas especializados llamados sistemas de procesamiento de \textit{stream} (SPS), los cuales pueden procesar en tiempo real los datos que van llegando por una o m\'as fuentes de datos. Estos sistemas est\'an basados en grafos, cuyos v\'ertices realizan operaciones sobre un flujo de datos reflejado por las aristas del grafo. La topolog\'ia del grafo le brinda flexibilidad al SPS para generar diversas aplicaciones de procesamiento, sin embargo, dicha topolog\'ia es est\'atica una vez el sistema se ejecuta. Dado este problema, este trabajo se plantea un modelo el\'astico que sea capaz de adaptar la topolog\'ia del grafo a las condiciones del tr\'afico existente. Para esto se ha dise\~nado un algoritmo reactivo, usando la t\'ecnica de fisi\'on, y otro predictivo, usando cadena de M\'arkov, ambas t\'ecnicas permiten estimar la carga de los operadores, y adaptar el grafo acorde a lo indicado por estos. Dicha modificaci\'on consiste en incrementar o disminuir la cantidad de r\'eplicas de un operador seg\'un su nivel de carga. Los resultados obtenidos de los experimentos realizados en el SPS S4, muestran una mejora de hasta nueve veces m\'as eventos procesados, con un costo asociado a un aumento de $0,01\%$ del uso de la CPU, pero una disminuci\'on de un $1,5\%$ en el consumo de memoria RAM.
\vspace*{0.5cm}
\KeywordsES{SPS; Elasticidad; Distribuci\'on de carga; Balance de carga; Algoritmos reactivos; Fisi\'on; Algoritmos predictivos; Cadena de M\'arkov}
}

\newpage

\resumenIngles{
Nowadays, information generated by the Internet's interactions is growing exponentially, creating massive and continuous flows of events from the most diverse sources. These interactions contain valuable information for domains such as government, commerce, and banks, among others. Extracting information from such data requires powerful processing tools to cope with the high-velocity and high-volume stream of events with near real-time results. Specially-designed distributed processing engines build a graph-based topology of a static number of processing operators creating bottlenecks and load balance problems when processing dynamic flows of events. In this work we propose a self-adaptive processing graph that provides elasticity and scalability, increasing or decreasing automatically the number of processing operators to improve performance and resource utilization.
Our solution uses a model that monitors, analyzes and changes the topology of the graph with a control algorithm that is both reactive and proactive to the flow of events.
We have compared our solution with a baseline approach and results show that our system  improves performance in terms of the number of processed events at a very low cost.
\vspace*{0.5cm}
\KeywordsEN{SPS; Elastic; Load balancing; Reactive Algorithm; Fision; Predictive Algorithm; Markov chain}
}
