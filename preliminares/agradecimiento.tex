\begin{agradecimiento}
Me cuesta encontrar las palabras para expresar lo que siento, se cierra un ciclo en mi vida, y con ello una larga travesía de mucho esfuerzo y trabajo, donde espero que de inicio a muchos más. Recibí apoyo, cariño y paciencia de diferentes personas, las cuales subían y bajaban de este excéntrico tren en cada parada que hacíamos, y aunque a veces entre paradas se hacía un viaje muy corto, habían otras veces que se hacían unos viajes eternos, y no terminaba de distinguir realmente bien dónde iba, pero eso daba igual, porque siempre en cada trayecto aprendía y recibía algo de alguna persona, y eso me hizo ir creciendo con el tiempo, lo cual estoy eternamente agradecido. De antemano pido disculpas si me he olvidado de alguien, no fue mi intención, y espero que pueda agradecerlo en un tiempo futuro. 

Aunque uno a veces siente que todo cambia; la gente, el barrio, la universidad, la sociedad, el trabajo, hay sentimientos y pasiones que nunca cambian, y son esas que te hacen vibrar y llenarte el corazón por estar ahí. Sin duda alguna, Izquierda Libertaria es una mis grandes pasiones, donde a través del FeL me brindó una escuela de lucha para poder construir un pueblo digno y soberano. No saben como agradezco estar ahí y conocer a mis amigas y amigos que además son compañeras y compañeros de lucha: Thomi, Neto, Cachorro, Sussan, Andrés, Zarri, Cata, Pato, Cristián, Tuto, Nati, Joaco, Fofi, de corazón gracias por todo su apoyo incondicional. Y más todavía agradezco al FeL porque aquí conocí a mi amada polola, quien no sólo le agradezco su apoyo, sino también por su cariño, alegría y paciencia, y sin duda todo sería muy distinto sin ti, no sabes lo agradecido y feliz que soy de estar con vos. Te amo Cami.

Cuando uno va viajando, vas creciendo, vas madurando, te vas dando cuenta que puedes ir resolviendo problemas que antes se hacían imposibles, van dándose nuevas herramientas, nuevas habilidades. Por lo mismo es que estoy completamente agradecido de la oportunidad que me brindaron mis queridos profesores guías Erika y Nicolás, porque confiaron en mí y se la jugaron por sacar este trabajo, y no sólo eso, de abrirme puertas para poder sustentarme y tener un porvenir más tranquilo. De verdad no saben lo orgulloso que estoy de tenerlos como profesores, no sólo me han ayudado a formarme como profesional, sino también como persona y eso se los agradezco desde el fondo de mi corazón. También agradezco a Pamela, por su apoyo y entrega incondiconal, y la profesora Carolina y el profesor Mauricio, que gracias a CITIAPS me sentí en mi segundo hogar, y siento que mucha de las cosas que aprendí hoy en día es gracias a esa pequeña salita de gran corazón. Y por lo mismo, no puedo olvidar a mi compañero de trabajo, Pablo, siempre me acuerdo como nos conocimos en ese primer día de las clases de Magíster, nunca creí que gracias a ese saludo pudiéramos tener este viaje, donde salió un paper, un proyecto y una bonita amistad, de verdad gracias Pablo, sos un hermano para mí. Gracias a todos los que pasaron y están en CITIAPS, Álvaro, Jeff, Diego, Farisori, no me olvidaré de ustedes, porque sin duda me dejaron marcado y me ayudaron a formarme como profesional y persona. También a mis amigas y amigos de la Usach, Miguel, Gabo, Clau, Karla, Jorge, los quiero caleta y gracias por todo, porque fueron un pilar en mi vida universitaria, donde no los olvidaré, porque todo lo que viví con ustedes fue una de mis grandes alegrías.

Pero a veces uno se queda dormido, recuerdas el pasado, y te acuerda que eso fue lo que te formó y lo que explica el cómo eres hoy en día. Nunca los voy a olvidar, para mi son mi fuerza, mis ganas, mi apañe, quizá todos estamos en nuestras vidas, pero los tengo mas presente que nunca, José, Eduardo, Rubén y Felipe, los quiero demasiado. Y como te voy a olvidar, si sos mi hermano de leche, Simón, no sólo te tengo que agradecer, te debería hacer una oda por lo grande que has sido conmigo. Tantas cosas que pasamos, tanto que vivimos, tantos recuerdos, y como fuimos creciendo, donde sea que estés tengo claro que puedo contar contigo, y estoy seguro que tú sientes lo mismo de mí.

Y alguien tuvo que crear la primera estación del viaje, alguien tuvo que armar ese tren, y es algo que aunque suba y bajen mil personas, pasen mil años, nunca podrás borrarlo, y me alegro que sea así, porque es lo más valioso que uno tiene, la familia. Como no quererlas, si me mimaron, me criaron, me abrazaron, me apoyaron, y eso lo agradezco, a ti Amandi, a ti Sofi y a ti Mamá, las amo. Obvio, no me voy a olvidar de ti, gracias por todo tu apoyo Papá, siento que mucho de lo que aprendí fue gracias a ti, y me alegro haber tenido esa oportunidad.
\end{agradecimiento}